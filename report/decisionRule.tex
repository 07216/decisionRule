\documentclass{article}
\usepackage{amsmath,amssymb,bbm}
\title{Linear Decision Rule Approach}
\author{Zhan Lin}

%\renewcommand{\arraystretch}{1.5}
\linespread{1.6}
\date{}
\begin{document}
\maketitle
\section{Original Problem}
Consider a network system consisting of $m$ resources, with capacity levels $\mathbf{c}=(c_1,\ldots,c_m)^T$, and $n$ products, with corresponding prices denoted by $\mathbf{v} = (v_1,\ldots,v_n)^T$. Each products needs at most one unit of each resource. Let $A = \left(a_{ij}\right)$ be the resource coefficient matrix, where $a_{ij}=1$ if product j uses one unit of resource i and $a_{ij}=0$ otherwise. Define $\xi = \left(1, \xi_{1,1}, \xi_{1,2}, \ldots, \xi_{1,n}, \ldots, \xi_{t,1}, \xi_{t,2}, \ldots, \xi_{t,n} \right)^T $ where $\xi_{t,j}$ is demand of product $j$ in period $t$. Let $\xi^t$ be the observed history demands, $\xi_t$ be the demand at period t and $$\mathbf{x_t}(\xi^t,\xi_t) = \left(x_{t,1}(\xi^t,\xi_{t,1}),x_{t,2}(\xi^t,\xi_{t,2}),\ldots,x_{t,n}(\xi^t,\xi_{t,n})\right)^T$$ be the booking limits in period t. Realisation of $\xi$ is limited to $\Xi$. The optimality equations can be expressed as

\begin{equation}
\begin{array}{ll}
\max &\mathbb{E}_{\xi}\left(\sum^T_{t=1} \mathbf{v}^T \mathbf{x_t} \left(\xi^t,\xi_t\right)\right)\\
s.t. & \sum^T_{t=1} A \mathbf{x_t} \left(\xi^t,\xi_t\right) \leq \mathbf{c}\\
& \mathbf{x_t}(\xi^t,\xi_t) \leq \xi_t\\
&\forall \xi \in \Xi,t=1,\ldots,T
\end{array}
\label{origin}
\end{equation}
\section{Linear Decision Rule Approach}
\subsection{Primal Problem}
Approach original problem with $$\mathbf{x_{t,j}}(\xi^t,\xi_{t,j}) = X_t \xi^t + \mathbf{\tilde{X}_t}^T\left(\mathbf{1}_{\xi \in (l_1,r_1)},\mathbf{1}_{\xi \in (l_2,r_2)},\mathbf{1}_{\xi \in (l_3,r_3)},\ldots,\mathbf{1}_{\xi \in (l_k,r_k)}\right)^T \xi_t $$ , $\Xi = \left\{ \xi : W \xi \leq h\right\}$, and $p_t \xi =\xi_t$ where $X_t$, $P_t$ , $W$ and $p_t$ are all matrix. Then we obtain the primal problem


\begin{equation}
\begin{array}{ll}
\max &\mathbb{E}_{\xi}\left(\sum^T_{t=1} \mathbf{v}^T X_t P_t \xi\right)\\
s.t. & \sum^T_{t=1} A X_t P_t \xi \leq \mathbf{c}\\
& X_t P_t \xi \leq p_t\xi\\
&\forall \xi \in \Xi = \left\{ \xi : W \xi \leq h\right\},t=1,\ldots,T
\end{array}
\label{primal}
\end{equation}
where $$W=\left(\begin{array}{lllll}
1 & & &\\
-1 & & &\\
& 1 & &\\
& -1 & &\\
& & \vdots &\\
& & & 1\\
& & & -1
\end{array}
\right)
$$
and $$h=\left(1,-1,q_{1,1,sup},-q_{1,1,inf},q_{1,2,sup},-q_{1,2,inf},\ldots,q_{t,n,sup},-q_{t,n,inf}\right)^T$$ in which $q_{t,j,p}$ is the $p$ percentile of $\xi_{t,j}$.

There are $n m t \tau + 2 n m t + n^2 t^2$ variables and $m+2nt$ constraints in total.

\subsection{Duality}

Firstly we transform equation(\ref{origin}) into a tighter formulation.
\begin{equation}
\begin{array}{ll}
\max &\mathbb{E}_{\xi}\left(\sum^T_{t=1} \mathbf{v}^T \mathbf{x_t} \left(\xi^t\right)\right)\\
s.t. & \sum^T_{t=1} \tilde{A} \mathbf{x_t} \left(\xi^t\right) \leq \mathbf{ \tilde{c}}_t\left(\xi\right)\\
& \mathbf{x_t} \left(\xi^t\right) \geq 0\\
&\forall \xi \in \Xi,t=1,\ldots,T
\end{array}
\label{duality:origin}
\end{equation}
Then it has a duality.

\begin{equation}
\begin{array}{ll}
\min &\mathbb{E}_{\xi}\left(\sum^T_{t=1}  \mathbf{ \tilde{c}}_t\left(\xi\right)^T \mathbf{y_t} \left(\xi^t\right)\right)\\
s.t. & \sum^T_{t=1} \tilde{A}^T \mathbf{y_t} \left(\xi^t\right) \geq \mathbf{v}^T\\
& \mathbf{y_t} \left(\xi^t\right) \geq 0\\
&\forall \xi \in \Xi,t=1,\ldots,T
\end{array}
\end{equation}
We can apply same approach and obtain

\begin{equation}
\begin{array}{ll}
\min &\mathbb{E}_{\xi}\left(\sum^T_{t=1}  \left(\mathbf{c}^T,(p_t\xi)^T \right)\mathbf{y_t} \left(\xi^t\right)\right)\\
s.t. & \sum^T_{t=1} \tilde{A}^T Y_t P_t \xi \geq \mathbf{v}^T\\
& Y_t P_t \xi  \geq 0\\
&\forall \xi \in \Xi\left\{ \xi : W \xi \leq h\right\},t=1,\ldots,T
\end{array}
\end{equation}

\section{Numerical Results}
Rounding can matter a lot to benefits. Therefore, I tried genetic algorithm to get a not bad policy. However, it sometimes offer a
worse policy than simply ceil. 
What's more, please notice that all results obtained now are under constant policies.

\subsection{Parameters}

The upper bound and lower bound of $\xi$ in $\Xi$ play important roles in optimizing. With strict bounds, booking limits will be under rigorous limitations. Therefore we need to set lower bound of the primal problem a bit higher. In the first example of Re-Solving Stochastic Programming Models for Airline Revenue, upper bound = 80 percentile and lower bound = 60 percentile. In the second example, upper bound = 90 percentile and lower bound = 70 percentile. Results are given at table \ref{Summary}. The results of data from approximate linear programming are not listed here. But in most cases, comparing with approximate linear programming, linear decision rules will only result in 5 percentile loss.

\begin{table}[]
\centering
\begin{tabular}{|l|c|c|c|c|}
\hline
                                  & DLP-alloc & SLP-alloc & LDR & Wait-and-see \\ \hline
First Example without re-solving  & 401980    & 415410    &  419831   & 432730       \\ \hline
Second Example without re-solving & 583630    & 595620    &  601212   & 432730       \\ \hline
First Example with re-solving     & 409740    & 421894    &     & 623530       \\ \hline
Second Example with re-solving    & 594021    & 604859    &     & 623530       \\ \hline
\end{tabular}
\caption{Summary}
\label{Summary}
\end{table}



%����c��ʱ�������ʵ�������ͬ
\end{document}
