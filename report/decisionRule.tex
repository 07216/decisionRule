\documentclass{article}
\usepackage{amsmath,amssymb}
\title{Linear Decision Rule Approach}
\author{Zhan Lin}

\renewcommand{\arraystretch}{1.5}
\date{}
\begin{document}
\maketitle
\section{Original Problem}
\begin{equation}
\begin{array}{ll}
\max &\mathbb{E}_{\xi}\left(\sum^T_{t=1} \mathbf{v}^T \mathbf{x_t} \left(\xi^t\right)\right)\\
s.t. & \sum^T_{t=1} A \mathbf{x_t} \left(\xi^t\right) \leq \mathbf{c}\\
& x_t(\xi^t) \leq p_t(\xi)\\
&\forall \xi \in \Xi,t=1,\ldots,T
\end{array}
\label{origin}
\end{equation}
\section{Linear Decision Rule Approach}
\subsection{Primal Problem}
\begin{equation}
\begin{array}{ll}
\max &\mathbb{E}_{\xi}\left(\sum^T_{t=1} \mathbf{v}^T X_t P_t \xi\right)\\
s.t. & \sum^T_{t=1} A X_t P_t \xi \leq \mathbf{c}\\
& X_t P_t \xi \leq p_t^T\xi\\
&\forall \xi \in \Xi = \left\{ \xi : W \xi \leq h\right\},t=1,\ldots,T
\end{array}
\label{primal}
\end{equation}

time complexity

\subsection{Duality}

Firstly we transform equation(\ref{origin}) into a tighter formulation.
\begin{equation}
\begin{array}{ll}
\max &\mathbb{E}_{\xi}\left(\sum^T_{t=1} \mathbf{v}^T \mathbf{x_t} \left(\xi^t\right)\right)\\
s.t. & \sum^T_{t=1} \tilde{A} \mathbf{x_t} \left(\xi^t\right) \leq \mathbf{ \tilde{c}}_t\left(\xi\right)\\
& \mathbf{x_t} \left(\xi^t\right) \geq 0\\
&\forall \xi \in \Xi,t=1,\ldots,T
\end{array}
\label{duality:origin}
\end{equation}
Then it has a duality.

\begin{equation}
\begin{array}{ll}
\min &\mathbb{E}_{\xi}\left(\sum^T_{t=1}  \mathbf{ \tilde{c}}_t\left(\xi\right)^T \mathbf{y_t} \left(\xi^t\right)\right)\\
s.t. & \sum^T_{t=1} \tilde{A}^T \mathbf{y_t} \left(\xi^t\right) \geq \mathbf{v}^T\\
& \mathbf{y_t} \left(\xi^t\right) \geq 0\\
&\forall \xi \in \Xi,t=1,\ldots,T
\end{array}
\end{equation}

Also,we can use linear decision rule approach.

\begin{equation}
\begin{array}{ll}
\min &\mathbb{E}_{\xi}\left(\sum^T_{t=1}  \left(\mathbf{c}^T,p` \right)\mathbf{y_t} \left(\xi^t\right)\right)\\
s.t. & \sum^T_{t=1} \tilde{A}^T Y_t P_t \xi \geq \mathbf{v}^T\\
& Y_t P_t \xi  \geq 0\\
&\forall \xi \in \Xi\left\{ \xi : W \xi \leq h\right\},t=1,\ldots,T
\end{array}
\end{equation}

\section{Numerical Results}
\subsection{First Case in Re-solve}
\subsection{Second Case in Re-solve}
\subsection{Some Tricks}

%����c��ʱ�������ʵ�������ͬ
\end{document}
