\documentclass{article}
\usepackage{amsmath,amssymb,bbm,comment}
\title{Linear Decision Rule Approach}
\author{Zhan Lin}

%\renewcommand{\arraystretch}{1.5}
\linespread{1.6}
\date{}
\begin{document}
\maketitle
\section{Original Problem}
Consider a network system consisting of $m$ resources, with capacity levels $\mathbf{c}=(c_1,\ldots,c_m)^T$, and $n$ products, with corresponding prices denoted by $\mathbf{v} = (v_1,\ldots,v_n)^T$. Each products needs at most one unit of each resource. Let $A = \left(a_{ij}\right)$ be the resource coefficient matrix, where $a_{ij}=1$ if product j uses one unit of resource i and $a_{ij}=0$ otherwise. Define $\xi = \left(1, \xi_{1,1}, \xi_{1,2}, \ldots, \xi_{1,n}, \ldots, \xi_{t,1}, \xi_{t,2}, \ldots, \xi_{t,n} \right)^T $ where $\xi_{t,j}$ is demand of product $j$ in period $t$. Assuming observing $\tau$ periods of history,let $\xi^t$ be the observed history demands, $\xi_t$ be the demand at period t and $$\mathbf{x_t}(\xi^t,\xi_t) = \left(x_{t,1}(\xi^t,\xi_{t,1}),x_{t,2}(\xi^t,\xi_{t,2}),\ldots,x_{t,n}(\xi^t,\xi_{t,n})\right)^T$$ be the booking limits in period t. Realisation of $\xi$ is limited to $\Xi$. The optimality equations can be expressed as

\begin{equation}
\begin{array}{ll}
\max &\mathbb{E}_{\xi}\left(\sum^T_{t=1} \mathbf{v}^T \mathbf{x_t} \left(\xi^t,\xi_t\right)\right)\\
s.t. & \sum^T_{t=1} A \mathbf{x_t} \left(\xi^t,\xi_t\right) \leq \mathbf{c}\\
& \mathbf{x_t}(\xi^t,\xi_t) \leq \xi_t\\
& \mathbf{x_t}(\xi^t,\xi_t) \geq 0\\
&\forall \xi \in \Xi,t=1,\ldots,T
\end{array}
\label{origin}
\end{equation}

It's important to notice that the j-th element of $\mathbf{x}_t$ 

\section{Linear Decision Rule Approach via Lifting}
\subsection{Primal Problem}
Approach original problem with $$x_{t,j}(\xi^t,\xi_{t,j}) = \left(X_t L_{t}\left(\xi^t\right) \right)_j + \tilde{X}_{t,j}L_{t,j}\left(\xi_{t,j}\right)=(X_tP_tL(\xi))_j+\tilde{X}_{t,j}Q_{t,j}L(\xi)$$ where
$$L_{t,j}(x) = (L_{t,j,d}(x))_{r_{t,j}\times1}$$
with
$$
L_{t,j,d}(x)  =
\left\{
\begin{array}{ll}
x & r_{t,j} = 1\\
\min\{x,z_1^{t,j}\}& r_{t,j}>1,d=1\\
\max\{\min\{x,z_d^{t,j}\}-z^{t,j}_{d-1},0\} & r_{t,j}>1,d=2,\ldots,r_{t,j}-1\\
\max\{x-z_{d-1}^{t,j},0\}& r_{t,j}>1,d=r_{t,j}
\end{array}
\right.
$$
and
$$
L(\xi^t)=(L_{\tilde{t},j}(\xi_{\tilde{t},j}))_{\tau\times j,1}
$$
It is obvious that the linear retraction operator corresponding to $L_{t,j}$ is $$R_{t,j}(y)=\sum^{r_{t,j}}_{d=1} y_d=e^T y$$
According to \cite{GDR},
with $\Xi = \left\{ \xi : \xi_1=1,l_{t,j}\leq \xi_{t,j} \leq r_{t,j}\right\}$,
$$\text{conv}\Xi'=L(\Xi)=\left\{\xi' : \xi_1'=1,V_{t,j}^{-1}(1,\xi_{t,j}')_{(d+1)\times1}^T\geq 0\right\}$$
$$
V_{t,j}^{-1}=\begin{pmatrix}
\frac{z^{t,j}_1}{z^{t,j}_1-l_{t,j}} & -\frac{1}{z^{t,j}_1-l^{t,j}_1}& & & &\\
-\frac{l_{t,j}}{z^{t,j}_1-l_{t,j}}& \frac{1}{z^{t,j}_1-l_{t,j}}& -\frac{1}{z^{t,j}_2-z^{t,j}_1}& & &\\
& &\frac{1}{z^{t,j}_2-z^{t,j}_1}&\ddots & &\\
& & &\ddots& -\frac{1}{z^{t,j}_{r_{t,j}-1}-z^{t,j}_{r_{t,j}-2}}&\\
& & & &\frac{1}{z^{t,j}_{r{t,j}-1}-z^{t,j}_{r_{t,j}-2} }&-\frac{1}{u_{t,j}-z^{t,j}_{r_{t,j}-1}}\\
& & & & & \frac{1}{u_{t,j}-z^{t,j}_{r_{t,j}-1}}
\end{pmatrix}
$$
Then we obtain the primal problem
\begin{equation}
\begin{array}{ll}
\max &\mathbb{E}_{\xi}\left(\sum^T_{t=1} \mathbf{v}^T (X_t P_t + \sum_{k=1}^j\tilde{X}_{t,k}Q_{t,k})\xi\right)\\
\text{s.t.} & \sum^T_{t=1} A (X_t P_t + \sum_{k=1}^j\tilde{X}_{t,k}Q_{t,k})\xi \leq 2\mathbf{c} - \mathbb{E}_{\xi}\left(\sum^T_{t=1} A (X_t P_t + \sum_{k=1}^j\tilde{X}_{t,k}Q_{t,k})\xi\right)\\
& (X_t P_t + \sum_{k=1}^j\tilde{X}_{t,k}Q_{t,k})\xi \leq e^T p_t\xi\\
& (X_t P_t + \sum_{k=1}^j\tilde{X}_{t,k}Q_{t,k})\xi \geq 0\\
&\forall \xi \in \text{conv}\Xi' ,t=1,\ldots,T
\end{array}
\label{primal}
\end{equation}

The correction $\mathbf{c} - \mathbb{E}_{\xi}\left(\sum^T_{t=1} A (X_t P_t + \sum_{k=1}^j\tilde{X}_{t,k}Q_{t,k})\xi\right) $arises from inaccurately approaching $\mathbf{x}(\xi^t,\xi_t)$.

There are $O(n m rt \tau + n^2 t^2 r )$ variables and $O(t^2 n^2 r+t n m r)$ constraints in total.
\begin{comment}
\subsection{Duality}

Firstly we transform equation(\ref{origin}) into a tighter formulation.
\begin{equation}
\begin{array}{ll}
\max &\mathbb{E}_{\xi}\left(\sum^T_{t=1} \mathbf{v}^T \mathbf{x_t} \left(\xi^t\right)\right)\\
s.t. & \sum^T_{t=1} \tilde{A} \mathbf{x_t} \left(\xi^t\right) \leq \mathbf{ \tilde{c}}_t\left(\xi\right)\\
& \mathbf{x_t} \left(\xi^t\right) \geq 0\\
&\forall \xi \in \Xi,t=1,\ldots,T
\end{array}
\label{duality:origin}
\end{equation}
Then it has a duality.

\begin{equation}
\begin{array}{ll}
\min &\mathbb{E}_{\xi}\left(\sum^T_{t=1}  \mathbf{ \tilde{c}}_t\left(\xi\right)^T \mathbf{y_t} \left(\xi^t\right)\right)\\
s.t. & \sum^T_{t=1} \tilde{A}^T \mathbf{y_t} \left(\xi^t\right) \geq \mathbf{v}^T\\
& \mathbf{y_t} \left(\xi^t\right) \geq 0\\
&\forall \xi \in \Xi,t=1,\ldots,T
\end{array}
\end{equation}
We can apply same approach and obtain

\begin{equation}
\begin{array}{ll}
\min &\mathbb{E}_{\xi}\left(\sum^T_{t=1}  \left(\mathbf{c}^T,(p_t\xi)^T \right)\mathbf{y_t} \left(\xi^t\right)\right)\\
s.t. & \sum^T_{t=1} \tilde{A}^T Y_t P_t \xi \geq \mathbf{v}^T\\
& Y_t P_t \xi  \geq 0\\
&\forall \xi \in \Xi\left\{ \xi : W \xi \leq h\right\},t=1,\ldots,T
\end{array}
\end{equation}
\end{comment}
\section{Computational Results}
As we can see from Preservation of Structural Properties in Optimization with Decisions Truncated by Random Variables and Its Applications, $\tilde{X}_{t,j}(\xi^t,\xi_{t,j})$ should behave $u\bigwedge \xi_{t,j}$. Benefiting from piecewise linear function, the structure is preserved with linear decision rule. Therefore, it's convenient to implement the policy as we can get $u$ from $\tilde{X}_{t,j}$.
What's more, rounding can matter a lot to benefits and some simple rounding policy is tested including ceiling, flooring, rounding, floor(x)+1. Results are computed at servers locating at USTC with 96 Intel Xeon CPU E5-4657L v2 2.4GHz and 256GB memory.Gurobi will make use of 32 cpu to solve the LP problem parallelly.
\begin{comment}
\subsection{Parameters}

The upper bound and lower bound of $\xi$ in $\Xi$ play important roles in optimizing. With strict bounds, booking limits will be under rigorous limitations. Therefore we need to set lower bound of the primal problem a bit higher. In the first example of Re-Solving Stochastic Programming Models for Airline Revenue, upper bound = 80 percentile and lower bound = 60 percentile. In the second example, upper bound = 90 percentile and lower bound = 70 percentile. Results are given at table \ref{Summary}. The results of data from approximate linear programming are not listed here. But in most cases, comparing with approximate linear programming, linear decision rules will only result in 5 percentile loss.
\end{comment}
\begin{table}[]
\centering
\begin{tabular}{|l|c|c|c|c|}
\hline
                                  & DLP-alloc & SLP-alloc & round & ceil  \\ \hline
Example 1 without re-solving  & 401980    & 415410    &  411651  & 421684       \\ \hline
Example 2 without re-solving & 583630    & 595620    &   590245  & 604264    \\ \hline
Example 1 with re-solving     & 409740    & 421894    &     &        \\ \hline
Example 2 with re-solving    & 594021    & 604859    &     &   \\ \hline
\end{tabular}
\caption{Summary}
\begin{tabular}{|l|c|c|c|}
\hline
                                  & floor &floor(x)+1&Wait-and-see\\ \hline
Example 1 without re-solving  &374129&422066&432730    \\ \hline
Example 2 without re-solving &560178&591747&623530       \\ \hline

Example 1 with re-solving     &  &     &      432730       \\ \hline
Example 2 with re-solving    & &     &      623530 \\ \hline
\end{tabular}
\caption{Summary}
\label{Summary}
\end{table}

\begin{table}[]
\centering
\begin{tabular}{|l|c|c|c|c|}
\hline
Case&Parameters& Computational Time/s&Optimal Objective&Best Simulation Result \\
\hline
1&$t=10,\tau=0,r=7$& 22.25&415305&421684\\
\hline
2&$t=10,\tau=0,r=7$& 29.49&596130&604264\\
\hline
2&$t=10,\tau=5,r=7$& 27280.45&599866&601999\\
\hline
2&$t=5,\tau=5,r=5$& 18.94&557856&562202\\
\hline
\end{tabular}
\caption{Summary}
\label{Summary}
\end{table}

\begin{thebibliography}{13}
\bibitem{GDR}Georghiou, A., Wiesemann, W. and Kuhn, D., 2015. Generalized decision rule approximations for stochastic programming via liftings. Mathematical Programming, 152(1-2), pp.301-338.
\end{thebibliography}

%����c��ʱ�������ʵ�������ͬ
\end{document}
